\documentclass[10pt,a4paper]{article}

% =========================================================
% CONFIGURACIÓN (ÚNICO BLOQUE EDITABLE)
% =========================================================
% --- Tipo de documento / layout ---
% true  => dos columnas (estilo “journal brief / one-pager / artículo”)
% false => una columna (mejor para propuestas/memorias extendidas)
\newif\ifDocTwoCol
\DocTwoColtrue

% Encabezado en páginas (recomendado en documentos extendidos):
\newif\ifDocUseHeader
\DocUseHeaderfalse

% Incluir teléfono (por defecto NO en artículos/one-pager):
\newif\ifIncludePhone
\IncludePhonefalse

% Año visible único en todo el documento (NO usar mes/día):
\newcommand{\DocYear}{2026}

% Título / subtítulo:
\newcommand{\DocTitle}{\textsc{TFM Co-diseño IA+FPGA}: Tensor-Train (TT) y kernel de contracción TT en FPGA}
\newcommand{\DocSubtitle}{ONE\_PAGER — Co-diseño IA+FPGA para capas lineales en Tensor-Train (TT)}
\newcommand{\DocShortTitle}{TFM TT en FPGA}

% Keywords (opcional):
\newcommand{\DocKeywords}{Tensor-Train; TT-SVD; FPGA; AXI; capas lineales}

% Idioma principal: seleccione UNO (spanish o english)
\newcommand{\MainLang}{spanish}

% --- Autor ---
\newcommand{\AuthorName}{Mariano Millañanco Fernández}
\newcommand{\AuthorLocation}{Ciudad Real, España}
\newcommand{\AuthorAffilA}{Universidad de Sevilla — Máster en Microelectrónica}
\newcommand{\AuthorAffilB}{UTAMED — Máster en Inteligencia Artificial}
\newcommand{\AuthorRepoUrl}{https://github.com/tangodelta217/TFM\_FPGA\_TT\_LLM\_INDRA}
\newcommand{\AuthorRepoText}{github.com/tangodelta217/TFM\_FPGA\_TT\_LLM\_INDRA}
\newcommand{\AuthorEmail}{mariano.millananco@gmail.com}
\newcommand{\AuthorPhone}{+34 638 97 15 46}
% Compatibilidad MiKTeX: evitar conflicto de \Bbbk con amssymb.
\AddToHook{package/amssymb/before}{\let\Bbbk\relax}
% =========================================================
% FIN CONFIGURACIÓN
% =========================================================

% --- Codificación e idiomas ---
\usepackage[utf8]{inputenc}
\usepackage[T1]{fontenc}
\usepackage[spanish,english,es-tabla]{babel}

% --- Márgenes (compactos pero seguros) ---
\usepackage[top=2.0cm,bottom=2.0cm,left=2.0cm,right=2.0cm]{geometry}

% --- Tipografía: cuerpo serif tipo Times + sans tipo Helvetica para títulos ---
\usepackage{newtxtext,newtxmath}
\usepackage[scaled=0.92]{helvet}

% --- Calidad tipográfica ---
\usepackage{microtype}

% --- Gráficos y tablas ---
\usepackage{graphicx}
\graphicspath{{figures/}{figuras/}}
\usepackage{booktabs}
\usepackage{tabularx,array}

% --- Matemáticas ---
\usepackage{amsmath,amssymb,mathtools}

% --- Listas compactas ---
\usepackage{enumitem}
\setlist[itemize]{leftmargin=*,noitemsep,topsep=0.25em}
\setlist[enumerate]{leftmargin=*,noitemsep,topsep=0.25em}

% --- Captions y floats ---
\usepackage{caption}
\usepackage{subcaption}
\usepackage{flafter}
\usepackage{placeins}
\usepackage{adjustbox}

% --- TikZ (si se usa, con estilos “safe” para evitar solapes) ---
\usepackage{tikz}
\usetikzlibrary{arrows.meta,positioning,shapes.geometric,calc}
\tikzset{
  safebox/.style={
    draw,
    rounded corners=2pt,
    align=center,
    font=\small,
    text width=0.26\linewidth,
    minimum height=10mm,
    inner sep=3pt
  },
  safearrow/.style={-Latex, line width=0.6pt},
  safedash/.style={dashed, line width=0.6pt}
}

% --- Color de acento (similar al ejemplo “excelso”) ---
\usepackage{xcolor}
\definecolor{Accent}{RGB}{0,70,130}
\definecolor{BoxBack}{RGB}{248,250,253}

% --- Estilo de secciones (visual premium) ---
\usepackage{titlesec}
\titleformat{\section}
  {\large\sffamily\bfseries\color{Accent}}
  {\thesection.}{0.6em}{}
\titleformat{\subsection}
  {\normalsize\sffamily\bfseries\color{Accent}}
  {\thesubsection}{0.6em}{}
\titlespacing*{\section}{0pt}{1.0ex plus 0.2ex minus 0.2ex}{0.6ex plus 0.2ex minus 0.2ex}
\titlespacing*{\subsection}{0pt}{0.9ex plus 0.2ex minus 0.2ex}{0.4ex plus 0.2ex minus 0.2ex}

% --- Caja ejecutiva tipo “Resumen ejecutivo” ---
\usepackage[most]{tcolorbox}
\newtcolorbox{ExecBox}[1]{%
  colback=BoxBack,
  colframe=Accent!35!black,
  boxrule=0.5pt,
  arc=2pt,
  left=6pt,right=6pt,top=5pt,bottom=5pt,
  fonttitle=\sffamily\bfseries,
  title=#1
}

% --- Espaciado de párrafos (evita “bloques amontonados”) ---
\setlength{\parindent}{0pt}
\setlength{\parskip}{0.35em}
\raggedbottom

% --- Separación de columnas (si se activa dos columnas) ---
\setlength{\columnsep}{0.6cm}

% --- Espaciado de floats/captions (compacto y estable) ---
\setlength{\textfloatsep}{0.9em plus 0.2em minus 0.2em}
\setlength{\intextsep}{0.9em plus 0.2em minus 0.2em}
\captionsetup{font=small,labelfont=bf}

% --- Anti overfull (sin recurrir a \sloppy) ---
\setlength{\emergencystretch}{2em}

% --- Header/footer (limpio) ---
\usepackage{fancyhdr}
\setlength{\headheight}{14pt}
\pagestyle{fancy}
\fancyhf{}
\ifDocUseHeader
  \fancyhead[L]{\small\sffamily\color{Accent}\DocShortTitle}
  \fancyhead[R]{\small\sffamily\color{Accent}\AuthorName}
  \renewcommand{\headrulewidth}{0.3pt}
  \renewcommand{\headrule}{\hbox to\headwidth{\color{Accent}\leaders\hrule height \headrulewidth\hfill}}
\else
  \renewcommand{\headrulewidth}{0pt}
\fi
\fancyfoot[C]{\sffamily\thepage}

% --- URLs/links robustos (evita desbordes por URLs largas) ---
\usepackage{xurl}
\usepackage{hyperref}
\usepackage{bookmark}
\hypersetup{
  unicode=true,
  colorlinks=true,
  linkcolor=Accent,
  citecolor=Accent,
  urlcolor=Accent,
  pdftitle={\DocTitle},
  pdfauthor={\AuthorName}
}
% Si el título usa \textsc, evitar problemas en metadatos PDF:
\pdfstringdefDisableCommands{%
  \def\textsc#1{#1}%
}

% --- Forzar “fecha visible” a solo año ---
\renewcommand{\today}{\DocYear}

% =========================================================
% BIBLIOGRAFÍA (EDITABLE: entradas reales, con DOI/URL)
% =========================================================
\usepackage{csquotes}
\usepackage{filecontents}
\begin{filecontents*}{references.bib}
@article{oseledets2011tt,
  author = {Oseledets, Ivan V.},
  title = {Tensor-Train Decomposition},
  journal = {SIAM Journal on Scientific Computing},
  volume = {33},
  number = {5},
  pages = {2295--2317},
  year = {2011},
  doi = {10.1137/090752286},
  url = {https://doi.org/10.1137/090752286}
}

@article{oseledets2009breaking,
  author = {Oseledets, Ivan V. and Tyrtyshnikov, Eugene E.},
  title = {Breaking the Curse of Dimensionality, or How to Use {SVD} in Many Dimensions},
  journal = {SIAM Journal on Scientific Computing},
  volume = {31},
  number = {5},
  pages = {3744--3759},
  year = {2009},
  doi = {10.1137/090748330},
  url = {https://doi.org/10.1137/090748330}
}

@article{kolda2009tensor,
  author = {Kolda, Tamara G. and Bader, Brett W.},
  title = {Tensor Decompositions and Applications},
  journal = {SIAM Review},
  volume = {51},
  number = {3},
  pages = {455--500},
  year = {2009},
  doi = {10.1137/07070111X},
  url = {https://doi.org/10.1137/07070111X}
}

@article{novikov2015tensorizing,
  author = {Novikov, Alexander and Podoprikhin, Dmitrii and Osokin, Anton and Vetrov, Dmitry},
  title = {Tensorizing Neural Networks},
  journal = {arXiv preprint},
  year = {2015},
  url = {https://arxiv.org/abs/1509.06569}
}

@article{garipov2016ultimate,
  author = {Garipov, Timur and Podoprikhin, Dmitrii and Novikov, Alexander and Vetrov, Dmitry},
  title = {Ultimate Tensorization: Compressing Convolutional and FC Layers Alike},
  journal = {arXiv preprint},
  year = {2016},
  url = {https://arxiv.org/abs/1611.03214}
}
\end{filecontents*}
\usepackage[backend=biber,style=ieee,sorting=none]{biblatex}
\addbibresource{references.bib}
\AtEveryBibitem{\clearfield{year}\clearfield{date}}
\AtBeginBibliography{\scriptsize}
\setlength{\bibitemsep}{0pt}

% =========================================================
% MACROS DE MAQUETACIÓN (NO EDITAR)
% =========================================================
\newcommand{\MakeTitleBlock}{%
  \begin{center}
    {\LARGE\bfseries \DocTitle\par}
    \vspace{0.25em}
    {\large \DocSubtitle\par}
    \vspace{0.85em}
    {\large\bfseries \AuthorName\par}
    {\normalsize \AuthorLocation\par}
    {\normalsize \AuthorAffilA\par}
    {\normalsize \AuthorAffilB\par}
    \vspace{0.25em}
    {\normalsize \href{\AuthorRepoUrl}{\AuthorRepoText}\par}
    {\normalsize \href{mailto:\AuthorEmail}{\AuthorEmail}%
      \ifIncludePhone\;|\;\AuthorPhone\fi\par}
    \vspace{0.65em}
    {\normalsize \DocYear\par}
  \end{center}
  \vspace{0.4em}
  {\color{Accent}\rule{\linewidth}{0.6pt}}
  \vspace{0.8em}
}

\newcommand{\FWHeading}[1]{%
  {\sffamily\bfseries\color{Accent}#1}\par\vspace{0.25em}
}

\newcommand{\BeginDoc}[1]{%
  \ifDocTwoCol
    \twocolumn[#1]
  \else
    #1
  \fi
}

% =========================================================
% DOCUMENTO
% =========================================================
\begin{document}

% Selección de idioma principal:
\ifx\MainLang\undefined\selectlanguage{spanish}\else
  \selectlanguage{\MainLang}
\fi

\BeginDoc{%
  \MakeTitleBlock

  % ===== Resumen / Abstract (top full-width) =====
  % Regla: ONE_PAGER => cada uno 60–90 palabras. Extendido => 120–180 palabras.
  \FWHeading{Resumen}
Se desarrolla un TFM de co-dise?o IA+FPGA para acelerar capas lineales mediante compresi?n Tensor-Train (TT) y un kernel de contracci?n TT en FPGA. El estado actual incluye demo CPU reproducible con TT-SVD, m?tricas de compresi?n y error, y un skeleton HLS/RTL con interfaces AXI y mapa de registros. La evaluaci?n usa golden model en Python y define KPIs HW como objetivos (TBD).

\FWHeading{Abstract}
This TFM targets IA+FPGA co-design to accelerate linear layers through Tensor-Train (TT) compression and an FPGA TT contraction kernel. Current status includes a reproducible CPU demo with TT-SVD, compression and error metrics, and an HLS/RTL skeleton with AXI interfaces and a register map. Evaluation uses a Python golden model and defines hardware KPIs as targets (TBD) for the FPGA phase.

% Keywords (opcional):
{\small\sffamily\textbf{Keywords:} \DocKeywords\par}
\vspace{0.6em}

% ===== Caja ejecutiva (recomendada en one-pager y útil en extendidos) =====
\begin{ExecBox}{Resumen ejecutivo (entregables medibles)}
  \begin{itemize}
    \item Demo CPU reproducible (TT-SVD y comparaci?n denso vs TT) con m?tricas en \texttt{docs/assets/kpi\_table.md} y \texttt{docs/assets/demo\_output.txt}.
    \item Skeleton HW con AXI, mapa de registros y diagrama de bloques; V\&V con golden model y tests; KPIs HW como objetivos (TBD).
  \end{itemize}
\end{ExecBox}
\vspace{0.6em}
}

% ============================
% CUERPO (generar según TIPO)
% ============================

% Sugerencia de estructura:
% - ONE_PAGER: 3–5 secciones cortas (puede ser IMRaD abreviado).
% - ARTICULO_EXTENDIDO: IMRaD completo.
% - PROPUESTA_TFM: motivación/objetivos/estado del arte/metodología/plan/riesgos/evaluación/resultados esperados.
% - MEMORIA_TFM: incluir ToC al inicio (solo si DocTwoColfalse) y secciones extensas.

\section{Introducción}
Las capas lineales en inferencia en borde son bandwidth-bound y el tráfico a DDR domina la latencia, afectando SWaP / determinismo / soberanía. El TFM aborda la reducción de tráfico mediante compresión Tensor-Train (TT) y un kernel de contracción TT en FPGA con streaming y control AXI.

\section{Metodología y arquitectura}
La operación base es $y = W x$, donde $W$ se aproxima en Tensor-Train (TT) con cores $\mathcal{G}_k \in \mathbb{R}^{r_{k-1}\times n^{out}_k\times n^{in}_k\times r_k}$. La descomposición se obtiene con TT-SVD y se valida en CPU con la referencia NumPy.

El kernel HW se diseña con AXI4-Lite, AXI-Stream/DMA y buffers locales. El layout de cores y el mapa de registros están en \url{docs/assets/register_map.md}.

\section{Evidencia, plan y limitaciones}
Evidencia actual (CPU): compresi?n 7.53x--341.33x y error relativo L2 8.85e-01 a \, $\sim$ \, 3e-15; tiempos en la tabla KPI (mediana de 40 repeticiones). Ver \url{docs/assets/kpi_table.md}, \url{docs/assets/bench_tradeoff.png} y \url{docs/assets/demo_output.txt}.

Plan de evaluaci?n: comparaci?n TT vs denso, tests autom?ticos y co-simulaci?n HW/SW con contadores de cycles/stalls. Resultados esperados: reducci?n del tr?fico a DDR y mayor estabilidad temporal; KPIs HW de latencia, throughput, recursos y determinismo como objetivos (TBD). Limitaciones actuales: sin bitstream ni implementaci?n HLS/RTL funcional y m?tricas solo en host CPU.

\FloatBarrier
\nocite{*}
\printbibliography[heading=none]

% ============================
% Checklist (comentario)
% ============================
% [OK] Sin primera persona plural
% [OK] Fechas visibles solo 2026
% [OK] Sin solapes; sin overfull severos
% [OK] Figuras/tablas dentro de \linewidth/\columnwidth
% [OK] Bibliograf?a real con DOI/URL (si hay \cite{})
\end{document}
